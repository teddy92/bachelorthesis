\chapter{Einleitung}
\section{Hochschule Heilbronn}

\section{Untersuchte Algorithmen}

Die Bilddatenverarbeitung (BDV) gewinnt in unserem technologischem Zeitalter immer mehr an Bedeutung. 
Durch die Verbreitung der Automatisierung in den meisten industriellen und sozialen Umfeldern werden immer neue und effizientere Methoden zur Datenverarbeitung benötigt.
Eine der größten Herausforderungen bei Autonomen Systemen ist die Erfassung der Umgebung um Interaktionen mir dieser, oder kollisionsfreihe Bewegungen zu ermöglichen. 
Zu diesem Zweck werden unter anderem Kameras verwendet.
Diese haben den Vorteil dass sie für viele verschiedene Aufgaben einsetzbar sind da sie eine enorme Menge an Daten sammeln. 
Allerdings werden sehr komplexe Algorithmen benötigt um aus diesen Daten Informationen wie die Art oder Position eines Objekts zu ermitteln. 
Ein Ansatz hierzu ist die Bestimmung von Punkten im Bild die bei möglichst vielen Lichtverhältnissen und aus unterschiedlichen Blickwinkeln erkannt werden können.
Durch die Anordnung dieser lokalen Merkmale kann man das Objekt lokalisieren und mittels eines Abgleichs mit einer vorher erstellten Datenbank kann man ein Objekt wiedererkennen.

