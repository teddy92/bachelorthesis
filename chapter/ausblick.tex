\chapter{Ausblick}

Die Ergebnisse zeigen dass sich die untersuchten Algorithmen zufriedenstellend für die Klassifizierung von bekannten Objekten eignen.
Um diese aber mit genügender Genauigkeit zu lokalisieren sind zumindest SURF und ORB nicht ausreichend.
Auch mit SIFT wird in manchen Situationen die Orientierung der Objekte nicht genau erkannt.
Um die Detektion und die Lokalisierung der Objekte zu stabilisieren könnte ein Kalman-Filter implementiert werden.
Dieser verwendet ein Bewegungsmodell welches von der Natur der Objekte und/oder der Bewegung der Kamera abhängt.
Im Zusammenspiel mit den Messungen der Position durch den in dieser Arbeit beschriebenen Prozess und der zu erwartenden Ungenauigkeit schätzt der Filter die aktuelle Position der Objekte.
Dies hat den Vorteil dass das Ergebnis nicht nur von der aktuellen Messung sondern auch von den vorherigen abhängt und somit das Rauschen der Messungen kompensiert werden kann.

