%==========================
\chapter{Sperrvermerk}
%==========================
Der vorliegende Praxissemesterbericht beinhaltet vertrauliche Informationen der Fraunhofer-Gesellschaft zur Förderung der angewandten Forschung e. V. und darf Dritten- ausßer Mitarbeitern der Hochschule Heilbronn Technik, Wirtschaft und Informatik im Rahmen des hochschulinternen Prüfungsverfahrens - nicht zugänglich gemacht werden. Veröffentlichungen und Vervielfältigungen, auch nur auszugsweise, sind ohne ausdrückliche Genehmigung der Fraunhofer-Gesellschaft zur Förderung der angewandten Forschung e. V. untersagt.

\chapter{Einleitung}
\section{Vorstellung des Unternehmens}
Fraunhofer ist die größte Forschungsorganisation für anwendungsorientierte Forschung in Europa. Die Forschungsfelder richten sich nach den Bedürfnissen der Menschen:
Gesundheit, Sicherheit, Kommunikation, Mobilität, Energie und Umwelt.

Das Institut wurde mit dem Ziel gegründet mit anwendungsorientierter Forschung einen unmittelbaren Nutzen für Unternehmen und damit auch für die Gesellschaft zu leisten.

Heute betreibt die Fraunhofer-Gesellschaft mehr als 80 Forschungseinrichtungen, davon 67 Institute, an über 40 Standorten in Deutschland. Mit einem Forschungsvolumen von
etwa 2 Milliarden Euro beschäftigen sich hier rund 23000 Mitarbeiter überwiegend mit natur- oder ingeieurwissenschaftlicher Ausbildung. Etwa 30 Prozent der Aufwendungen erhält
die Gesellschaft als institutionelle Förderung von Bund und Ländern, um Vorlaufforschung zu betreiben.

Eines der 67 Institute ist das Fraunhofer \emph{IPA}. Mit annähernd 1000 Mitarbeitern zählt es zu einem der größten Institute der Fraunhofer-Gesellschaft. Das Jahresbudget
beträgt über 60 Millionen Euro, davon stammt mehr als ein Drittel aus Industrieprojekten.

\subsection{Geschäftsfelder}

Die 13 Fachabteilungen des Fraunhofer \emph{IPA} werden ergänzt von den sechs Geschäftsfeldern Automotive, Maschinen- und
Anlagebau, Elektronik- und Mikrosystemtechnik, Energiewirtschaft, Medizin- und Biotechnik sowie Prozessindustrie.
Mit dieser Struktur unterstützt das Fraunhofer \emph{IPA} seine Praxispartner dabei, ihre Marktposition zu verbessern
und begleitet deren Markteintritt in neue Anwendungsbereiche.


\begin{figure}[h]
    \centering
    \includegraphics[scale=1.6]{geschaeftsfelder}
   \caption{Geschäftsfelder des Fraunhofer \emph{IPA}}
    \label{img:geschaeftsfelder}
\end{figure}
\paragraph{Automotive}
Durch den systemisch-interdisziplinären Ansatz gilt das Fraunhofer \emph{IPA} in ausgewählten Kompetenzfeldern als renommierter Partner für die Automobilindustrie.
Die übergreifende Funktion des Geschäftsfeldes ermöglicht es, Kompetenzen verschiedener Abteilungen zu kombinieren.
Die Forschungsschwerpunkte sind:

\begin{itemize}
    \item Wandlungsfähige Produktion
    \item Multi-Material Verbundwerkstoffe in der Produktion
    \item Produktion für Elektromobilität
    \item Der assistierte Mensch in der automobilen Produktion
\end{itemize}

Ein besonderes Augenmerk wird auf die Themen Arbeitsschutz, Prozesssicherheit, Qualität, Rationalisierungspotentiale, Flexibilität und nachhaltige Ressourcenorientierung gerichtet.

So werden verschiedene Dienstleistungen über den gesamten Projektzyklus entlang der Wertschöpfungskette angeboten.
An den Schnittstellen zwischen den Anforderungen der Branche und den Kompetenzen der Abteilung werden die Chance, neue Produktideen und Projekte zu initiieren genutzt.
\paragraph{Elektronik und Mikrosystemtechnik}
Eine Vielzahl innovativer Hightech-Produkte ist nur aufgrund elektronischer Systeme und Mikrosystemtechnik realisierbar.
Ob dies Stehts für die mikromechanische Aufweitung von Blutgefäßen, ABS-Bremsaggregate mit hohen Anforderungen an die Mikrofluidik, Zahnimplantate mit großer mechanischer Stabilität,
Materialauswahl, hygienischer und reinheitstechnischer Eigenschaften oder auch die immer kleiner werdenden elektronischen Systeme mit kontinuierlich steigenden Leistungsdichten sind:
sie alle stellen höchste Anforderungen an die Fertigungs- und Produktionstechnik.

Zudem stehen diese Hightech-Industriezweige unter einem extremen Kostendruck bei gleichzeitig sehr kurzen Markteinführungszeiten, so dass sie zum Einsatz innovativer und alternativer
Herstellungs- und Automatisierungslösungen gezwungen sind.
\paragraph{Energiewirtschaft}
Das Fraunhofer IPA arbeitet seit vielen Jahren an effizienzsteigernden Lösungen und Planungsmethoden in den Bereichen Oberflächentechnik, Fabrikgestaltung und Roboter- und Maschinentechnik.
»Smart Energy in Production« behandelt im Geschäftsfeld Energiewirtschaft das Zukunftsthema »Micro Smart Grid« und energieflexible Produktion mit der Fokussierung auf industrielle Nutzbarkeit.

Neben der Nutzung von Einsparpotenzialen und der intelligenten Energieverteilung versprechen Neuentwicklungen aus dem Bereich der Speichertechnologien eine Vielzahl von Einsatzmöglichkeiten
in der Intralogistik, Mobilität und bei Consumer-Produkten. So werden Batterien mit hohen Energiedichten und Superkondensatoren mit hohen Leistungsdichten vermehrt bei der Energierückgewinnung
bzw. Rekuperation in der Antriebstechnik eingesetzt. Ein sehr großes Potenzial wird dabei der intelligenten Kombination von Superkondensatoren und Batterien zugeschrieben.
\paragraph{Maschinen- und Anlagenbau}
Der Maschinen- und Anlagenbau ist Deutschlands größter Arbeitgeber mit mittelständischen Strukturen und führender Innovationskraft.
Seit über 50 Jahren arbeitet das Fraunhofer \emph{IPA} mit Unternehmen aus der Branche partnerschaftlich zusammen. Dabei unterstützt das Geschäftsfeld »Maschinen- und Anlagenbau«
in den Bereichen Fabrik- und Produktionsorganisation, Produktionstechnik und Automatisierung sowie Prozess- und Verfahrensentwicklung.

Durch die Herausforderungen der Branche, wie z. B. die hohe Marktdynamik, neue Technologien oder die Ressourcenverknappung, sehen wir vier strategische Entwicklungsfelder,
die den Weg zu einer »Smart Factory« ebnen:

\begin{itemize}
    \item Die Entwicklung neuer Produktionstechnik beinhaltet neue Materialien, die Schaffung neuer Automatisierungspotenziale sowie die Automatisierung in neuen Anwendungsfeldern.
      Die Produktion und ihre Mitarbeiter werden dabei zunehmend durch technische Assistenzsysteme unterstützt.
    \item Industrie 4.0 trägt dazu bei, die Intelligenz in Produktionssystemen zu erhöhen und Produkte sowie deren Produktion zu optimieren.
      Dies erfordert neue IT-Architekturen und -Services wie auch neue Organisationsmethoden und -prozesse.  Durch die Vernetzung der physischen
      und digitalen Produktion und durch die Integration der beiden in IT-Systeme, wird die Vision Industrie 4.0 Realität.
    \item Eine ressourceneffiziente Produktion wird durch systematisches Energie- und Materialmanagement erreicht. Hier sind nicht nur Kosteneinsparungen möglich,
          sondern auch die Leistungsfähigkeit wird erhöht.
    \item Wandlungsfähigkeit ist überall notwendig, wo hohe Marktflexibilität gefordert ist. Mit wandlungsfähigen Fabriken, modularen Produktionssystemen oder auch modularen IT-Architekturen
      und Schnittstellentechnik bringen Unternehmen ihre Produktentwicklungen schneller an den Markt.
\end{itemize}

\paragraph{Medizin- und Biotechnik}
Das Fraunhofer \emph{IPA} blickt im Bereich der Medizin- und Biotechnik auf jahrzehntelange Erfahrung zurück. Unser Dienstleistungsportfolio umfasst Beratungsleistungen, die Entwicklung von Instrumenten,
Geräten und Anlagen sowie Technologie- und Verfahrensentwicklungen bzw. -modifikationen.
\begin{itemize}
    \item Medizintechnik
    \item Biotech \& Pharma
    \item Diagnostik \& Intervention in der Klinik
    \item Quality \& Regulatory Affairs
    \item Reinheit in Life-Science-Branchen
    \item Produktions- und Prozessoptimierung
\end{itemize}

\paragraph{Prozessindustrie}
Die Prozessindustrie ist dadurch charakterisiert, dass Herstellungs- und Wertschöpfungsprozesse kontinuierlich und mit fließenden Materialien oder Medien ablaufen,
und bildet das Gegenstück zur Stückgutindustrie. Einzelne Produktionsschritte werden oft aufeinanderfolgend durchgeführt, sodass die Produkte oder die Zwischenprodukte
sich in Reaktoren befinden und in Rohrleitungen kontinuierlich von Station zu Station transportiert werden.

Das Fraunhofer \emph{IPA} verfügt über profunde Expertise im Bereich der Prozessindustrie und bildet diese als Geschäftsfeld im Dienstleistungsportfolio ab.
Dieses Geschäftsfeld bündelt Kompetenzen aus acht Abteilungen und unterstützt Unternehmen aus diesem Bereich als unabhängiger und kompetenter Partner.
Im Fokus stehen dabei zunächst die chemische Industrie, die Pharma- und die Stahlindustrie.

\subsection{Fachabteilungen}
Aufgrund der Vielzahl von verschiedenen Fachabteilungen am Fraunhofer \emph{IPA} werden im Folgenden lediglich die für das Praktikum relevanten Fachabteilungen der Automatisierung vorgestellt.

\paragraph{Steuerungs- und Antriebstechnik}
Hochperformante Steuerungs- und Antriebstechnik im Grenzbereich des technisch machbaren kann den Unterschied machen, wenn es darum geht, ein Projekt erfolgreich zu akquirieren,
den Gewinn zu maximieren oder in Hochlohnländern zu produzieren. Dabei sind einerseits kontinuierliche Verbesserungen der Maschinen und Anlagen sowie andererseits auch die grundlegende
Erneuerung der Technologie die Voraussetzungen für den dauerhaften Erfolg.

Hierfür stellen wir uns täglich den unterschiedlichsten Herausforderungen im Bereich der Automatisierungstechnik. Mit unserem Leistungsangebot zur »Steuerungs- und Antriebstechnik« unterstützen wir Sie bei der Verfolgung der wesentlichen Produktionsziele im Bereich Automatisierungstechnik.
\paragraph{Roboter- und Assistenzsysteme}
Die Abteilung »Roboter- und Assistenzsysteme« gestaltet Roboter und Automatisierungslösungen für industrielle Anwendungen und für den Dienstleistungsbereich.
Schlüsseltechnologien werden entwickelt und in innovative Industrieroboter, Serviceroboter und intelligente Maschinen umgesetzt.

40 Jahre Erfahrung in der Robotik und Automatisierung, multidisziplinäre Teams, ein einzigartiges Netzwerk, umfassendes Know-how sowie bestens ausgestattete Labors und Werkstätten bündeln
sich im Spektrum unserer Dienstleistungen:
\begin{itemize}
    \item Systemkonzeption
    \item Machbarkeitsstudien
    \item Simulation von Roboteranlagen und Komponenten
    \item Materialflusssimulation
    \item Entwicklung von Prototypen
    \item Erstellung von Lasten- und Pflichtenheften
    \item Vermessung von Robotern und Anlagen
    \item Optimierung bestehender Systeme
\end{itemize}