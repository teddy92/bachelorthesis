%=====================
\chapter{Hilfreiche Erkenntnisse für die weitere Softwareentwicklung}\label{app_1}
%=====================
\section{Debugging}
\label{debug}
%=====================
Debugging ist einer der teuersten Schritte in der Softwareentwicklung und nimmt einen Großteil der Kosten ein, die die Entwicklung kosten.

Unter diesem Aspekt habe ich die im ROS-Framework implementierte \quotes{rosconsle} sehr zu schätzen gelernt.
Sie bietet ein umfangreiches Set an Nachrichten-Levels (\quotes{verbosity}).

\begin{table}
\begin{center}
\begin{tabular}{p{3cm}  p{10cm}}
\textbf{Level} & \textbf{Beschreibung} \\ \hline	
	DEBUG & Nachrichten, die man nie sehen muss, wenn das Programm ordnungsgemäß läuft  \\ \hline
	INFO & Kleine Mengen an Informationen, die für den Anwender nützlich sind \\ \hline
	WARN & Informationen, die den Nutzer beunruhigen und die Ausgabe des Programms verändern können, aber im erwarteten Rahmen der Programmfunktion liegen \\ \hline
	ERROR & Ein ernster Fehler ist aufgetreten, aber das System kann sich von dem Fehler wieder erholen \\ \hline
	FATAL & Ein ernster Fehler ist aufgetreten und die Funktion kann nicht wiederhergestellt werden \\ 

\end{tabular}
\end{center}
\caption{\label{debugVerbosity}In der Tabelle werden die verschieden Debug-Levels beschrieben }
\end{table}
Ein weiteren Vorteil bringen benannte Logger-Nachrichten. 
Damit kann man später Filter einrichten. 
Am besten sieht man es am Beispiel:

\begin{verbatim}
ROS_DEBUG_NAMED_STREAM("task.step","Value is:"<<x);
\end{verbatim}

