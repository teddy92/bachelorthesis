\chapter*{Thesen}
\thispagestyle{empty}

Die Thesen sollen die wesentlichen Aussagen der Diplomarbeit in pr�gnanter Form
darstellen.

\begin{itemize}
    \item Zur L�sung des St�rmerproblems der deutschen
    Fu�ball-Nationalmannschaft ist der Einsatz zuverl�ssiger humanoider
    Roboter dringend notwendig.
    \item Gegenw"artig existierende Konzepte und L"osungen sind nicht in der Lage,
    die komplexen Anforderungen, die das Szenario Fu�ballspiel an den Roboter
    stellt, zu erf"ullen.
    \item Das Verfahren ABC "uberwindet die Einschr"ankungen der Methode xyz und
    erm"oglicht einem Roboter die intuitive Erfassung der Spielsituation mittels
    optischer Sensorik. 
    \item Durch die Weiterentwicklung des Schuss-Systems RoboFoot werden
    deutlich h"ohere Ballgeschwindigkeiten erzielt, was zu einer Erh"ohung der
    durchschnittlichen Erfolgsrate beim Torschuss f"uhrt.
    \item Es wurde experimentell nachgewiesen, dass der Roboter-Fu�baller
    Robario seinen menschlichen Konkurrenten in der Chancenverwertung um ca. 112\%
    �berlegen ist.
\end{itemize}

  \vspace*{1cm}

  \begin{sloppy}
  \hspace{2em}\begin{tabular}{lp{10cm}}
    \multicolumn{2}{l}
    {Ilmenau, \finishdate \hspace{5.5cm} \dotfill}\\
    \multicolumn{2}{l}
    {\hspace{8,5cm}Max Mustermann} \\
  \end{tabular}
  \end{sloppy}
\vfill
