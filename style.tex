%%%Dokumentklasse
\documentclass[12pt,twoside,fleqn]{book}

\usepackage{graphicx}
\usepackage{wrapfig}
\graphicspath{ {img/} }	
\usepackage[a4paper,width=150mm,top=25mm,bottom=25mm]{geometry}
\usepackage{calc}
\usepackage{amsmath}
\usepackage{amssymb}
\usepackage{amsfonts}           % einige weitere Sonderzeichen
\usepackage{subfigure}          % erlaubt das Anordnen von Bildern nebeneinander mit einzelnen Bildunterschriften
                                % einfacher, als mit Tabellen und Boxen zu arbeiten
\usepackage{ngerman}
\usepackage{bibgerm}            % Stylefile für deutsche Literaturstellenangabe
\usepackage[T1]{fontenc}        % saubere Trennung von Wörtern mit Umlauten
\usepackage[utf8]{inputenc}   % direkte Eingabe von Umlauten
\usepackage{fancyhdr}
\usepackage{colortbl}
\usepackage{hhline}
\usepackage[bf]{caption}
\usepackage[noadjust]{cite}     % erlaubt Zeilenumbruch innerhalb von Zitierungen, Option noadjust verhindert
                                % automatische Leerzeichen um die Referenz, was am Zeilenanfang zu Problemen führt
\usepackage{hyperref}           % erlaubt die Nutzugn von \url
\usepackage{code}          % Style vom Fachgebiet NIKR für Pseudocode-Darstellung

%\usepackage{pdfpages}
\usepackage[xindy,toc]{glossaries}
\sloppy

\usepackage{placeins}
% === Längen und Abstände ==================================================

% horizontales Layout
\setlength{\oddsidemargin}{0.2in}
\setlength{\evensidemargin}{0.0in}
\setlength{\textwidth}{\paperwidth - 2.2in}

% vertikales Layout
%\setlength{\topskip}{0.0cm}
\setlength{\headheight}{15.1pt}
%\setlength{\headsep}{0.0cm}
\setlength{\topmargin}{0.0cm}
\setlength{\footskip}{0.6in}
\setlength{\textheight}{\paperheight - 2.0in}
\addtolength{\textheight}{-1.0\headheight}
\addtolength{\textheight}{-1.0\headsep}
\addtolength{\textheight}{-1.0\footskip}

% Zeilenabstand
\renewcommand{\baselinestretch}{1.5}

\setlength{\topsep}{0.3cm}
\setlength{\mathindent}{1.0cm}
\setlength{\parindent}{0.0cm}

% === Bildunterschrift ========================================================

\renewcommand{\captionfont}{\small}
\newcommand{\NIcaption}[2]{\caption[#1]{#1\protect\\ \emph{#2}}}
\setcaptionmargin{0.75cm}
%\renewcommand{\topfraction}{0.95}
%\renewcommand{\textfraction}{0.05}
%\renewcommand{\textfloatsep}{1.0cm}

% === Seitenstil ==============================================================

\pagestyle{fancy}
\fancyhead[ER]{\itshape\leftmark}
\fancyhead[OL]{\itshape\rightmark}
\fancyhead[EL,OR]{\thepage}
\fancyfoot[ER,OL]{\small \invNumText}
%\fancyfoot[ER,OR]{\invNumText}
%\fancyfoot[EL,OL]{\myName}
\fancyfoot[EC,OC]{}
\renewcommand{\headrulewidth}{1pt}
\renewcommand{\footrulewidth}{0.4pt}

\fancypagestyle{plain}{
\fancyhead[ER]{\itshape\leftmark}
\fancyhead[OL]{\itshape\rightmark}
\fancyhead[EL,OR]{\thepage}
\fancyfoot[ER,OL]{\small \invNumText}
%\fancyfoot[ER,OR]{\invNumText}
%\fancyfoot[EL,OL]{\myName}
\fancyfoot[EC,OC]{}
\renewcommand{\headrulewidth}{0.4pt}
\renewcommand{\footrulewidth}{0.4pt}}

\newcolumntype{C}{>{\centering\arraybackslash}p{5.5em}}

% von den nachfolgenden Bl�cken bitte den richtigen
% ausw�hlen, die anderen auskommentieren/l�schen

% -------- Diplom ----------
%\newcommand{\Degree}{{Diplom}}
%\newcommand{\DegreeName}{{Diplominformatiker}}     % je nach Fachrichtung ausw�hlen
%\newcommand{\DegreeName}{{Diplomingenieur}}
%\newcommand{\invNum}{{xxx-xxx-xxx}}                % wird vom Pr�fungsamt vergeben    
%\newcommand{\invNumText}{{Inv.-Nr: \invNum}}
%\newcommand{\invNumTextLong}{{Inventarisierungsnummer: \invNum}}
%\newcommand{\inputthesen}{\chapter*{Thesen}
\thispagestyle{empty}

Die Thesen sollen die wesentlichen Aussagen der Diplomarbeit in pr�gnanter Form
darstellen.

\begin{itemize}
    \item Zur L�sung des St�rmerproblems der deutschen
    Fu�ball-Nationalmannschaft ist der Einsatz zuverl�ssiger humanoider
    Roboter dringend notwendig.
    \item Gegenw"artig existierende Konzepte und L"osungen sind nicht in der Lage,
    die komplexen Anforderungen, die das Szenario Fu�ballspiel an den Roboter
    stellt, zu erf"ullen.
    \item Das Verfahren ABC "uberwindet die Einschr"ankungen der Methode xyz und
    erm"oglicht einem Roboter die intuitive Erfassung der Spielsituation mittels
    optischer Sensorik. 
    \item Durch die Weiterentwicklung des Schuss-Systems RoboFoot werden
    deutlich h"ohere Ballgeschwindigkeiten erzielt, was zu einer Erh"ohung der
    durchschnittlichen Erfolgsrate beim Torschuss f"uhrt.
    \item Es wurde experimentell nachgewiesen, dass der Roboter-Fu�baller
    Robario seinen menschlichen Konkurrenten in der Chancenverwertung um ca. 112\%
    �berlegen ist.
\end{itemize}

  \vspace*{1cm}

  \begin{sloppy}
  \hspace{2em}\begin{tabular}{lp{10cm}}
    \multicolumn{2}{l}
    {Ilmenau, \finishdate \hspace{5.5cm} \dotfill}\\
    \multicolumn{2}{l}
    {\hspace{8,5cm}Max Mustermann} \\
  \end{tabular}
  \end{sloppy}
\vfill
}

% -------- Bachelor ----------
\newcommand{\Degree}{{Bachelor}}
\newcommand{\DegreeName}{{Bachelor of Engineering}}
\newcommand{\invNumText}{{}}
\newcommand{\invNumTextLong}{{}}
\newcommand{\inputthesen}{}

% -------- Master ----------
%\newcommand{\Degree}{{Master}}
%\newcommand{\DegreeName}{{Master of Science
%\newcommand{\invNumText}{{}}
%\newcommand{\invNumTextLong}{{}}
%\newcommand{\inputthesen}{}

\newcommand{\myName}{{Elias Marks}}   % der eigene Name
\newcommand{\finishdate}{{14.03.2016}}   % Datum der Abgabe

\graphicspath{{img/}}                    % Suchpfad (Unterverzeichnis) f�r eingebundene Grafiken

